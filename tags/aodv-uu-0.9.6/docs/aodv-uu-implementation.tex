\documentclass[swedish,10pt,twocolumn]{article}

\usepackage{graphicx}                  
\usepackage[T1]{fontenc}
\usepackage[latin1]{inputenc}
\usepackage[english]{babel}
%\usepackage[english]{babel}
\usepackage{sectsty}
\usepackage[normalem]{ulem}
%\usepackage{a4dutch}
\usepackage{url}
\usepackage{footnpag}
\usepackage{fancyhdr}
\usepackage{pslatex}
\usepackage{makeidx}

\title{The AODV-UU Implementation}

\author{Erik Nordstr\"om}

\makeindex

\begin{document}

\maketitle

\tableofcontents

\section{Introduction}

AODV-UU is an AODV routing protocol implementation developed at
Uppsala University, Sweden. AODV-UU runs in user space on the Linux
operating system or in the ns-2\cite{ns-2} simulation environment. 

AODV differs in many respects from traditional routing algorithms by
incorporating reactive and on-demand properties in the
protocol. Because of this, AODV-UU has to manage soft state
information about routes and administrative protocol message exchanges
between nodes. These functionalities increases the complexity of the
protocol implementation as, among other things, the user space
component needs intricate knowledge of information only accessible
from within kernel space. Some of the areas which makes implementing
the AODV routing protocol different from traditional routing protocols
are discussed below.

\subsection{User Space Approach}


It will be clear that AODV has several properties that make it
attractive to implement AODV in kernel space. However, the modern
Linux kernel provides so many powerful features for doing work in user
space, that it would be a shame to not exploit them. User space
programming also has several advantages:

\begin{itemize}
\item Simplified coding - use libraries and common C network programming
\item Easily debug code
\item Better portabilty to other OS:es and simulator environments like ns-2
\end{itemize}

The disadvantage is speed. AODV-UU performs user space packet
processing at a speed penalty. However, this penalty has no real world
effect on applications or the evaluation of the protocol. The main
purpose of AODV-UU is to test the AODV routing algoritm, not to enable
mp3 streaming, file downloading or Quake deathmatches over ad-hoc
networks, although all this is possible.

\section{Implementation Challenges}

\subsection{Routing Tables}

AODV-UU is a routing protocol, and therefore manages routing
tables. AODV needs to keep lots of information in the routing table
that are not available in the kernel's IP routing table. To solve this
problem, AODV-UU manages its own routing table in user space. The
AODV-UU routing table and the kernel's routing table are always kept
synchronized, and the kernel routing table is strictly used for IP
route resolving once AODV-UU has had a chance to initate its own route
resolve routines.

The AODV-UU routing table and related management routines are defined
and implemented in {\tt routing\_table.h} and {\tt routing\_table.c}.

\subsection{IP Route Resolving Mechanism}

Normally, when a packet is about to be sent, the route resolve procedure
for IP is to examine the kernel internal routing table for a matching
entry for the IP address of the requested destination. If a matching
entry is found, the packet is sent to the next hop (or gateway) as
indicated by that entry. If there is no match, but there is an entry
matching the subnet of the sender, the routing code assumes that the
node is on a broadcast network (typically a LAN), and sends the packet
directly to the destination. As a last resort, the packet might be
sent to a default gateway, if such an entry exists in the routing
table.

If IP fails to find a matching route entry, a ``Network is unreachable''
message is returned. In the case there is a matching subnet entry, but
the host is not available on the local network, the Address Resolution
Protocol (ARP), which resolvs IP-to-Link Layer address mappings will
eventually generate ``Destination Host Unreachable'' ICMP messages to
the application requesting the destination. This will often lead to
the termination of the request.

\subsection{On-demand Route Discovery}

AODV discovers routes on demand. Therefore, AODV-UU needs to intercept
any packets to a destination for which there is no route, so that the
packets can be buffered, while a route request packet (RREQ) is
disseminated. If packets are not intercepted, and allowed to continue
their traversal of the networking stack, they will eventually generate
any of the above described error messages, which can be fatal for
connection oriented protocols, like TCP.

\subsection{Routing Table Soft State}

In AODV, routing table entries are only valid as long as they are being used,
i.e. as long as there is traffic being forwarded. To facilitate this
functionality, AODV-UU needs to monitor the packet flow, so that route
utilization records can be kept. Whenever packets have not been sent
along a route for a defined period of time, the route entry should be
invalidated. This is to make sure AODV does not keep stale routing
table information.

\section{The Linux Netfilter Framework}

AODV-UU uses the Linux Netfilter framework to be able to manage the
intricate interaction with the packet flow of the networking
stack. Netfilter is native to the Linux kernel v2.4 and provides the
ability to perform advanced packet filtering and mangling, typically
to enable Linux to act as a firewall. Netfilter provides well defined
hooks in the kernel's networking stack. At these points, custom
intermediate code segments, which have the ability to drop, overtake
or change packets, can be inserted. 

\subsection{AODV-UU's Netfilter Kernel Module}
\index{kaodv.c}

AODV-UU has the {\tt kaodv.c} kernel module code to hook into the
kernel's networking stack at netfilter defined locations. All network
interface inbound packets or packets generated locally by applications
are redirected to this ``kaodv'' code. Packets which are ``captured''
are at this point queued for processing in user space, by means of
another Netfilter functionality (ip\_queue). Using a so called Netlink
socket, either the whole packet or just parts of packets
(i.e. headers) are sent to user space, so that a decision on what to
do with the packet can be taken. A verdict for the packet is returned
on the netlink socket, once the user space packet processing code has
finished.

\subsection{Route Discovery}
\index{packet\_input()}
\index{packet\_input.c}
\index{icmp.c}

In the AODV-UU case, data packets are sent to user space by the
``kaodv'' kernel module and received by the function {\tt
packet\_input()} in {\tt packet\_input.c}. In this code, route
utilization records can be updated, and a route discovery procedure
can be initiated if there is no destination entry in AODV-UU's
internal routing table. If a route discovery fails, the application
initiating the route discovery is notified via an ICMP message. ICMP
code is implemented in {\tt icmp.c}.

\subsubsection{Packet Buffering}
\index{packet\_queue()}
\index{packet\_queue.c}

In case AODV-UU needs to initiate a route discovery, pending data
packets for the sought destination should be buffered. AODV-UU
implements a packet queue in {\tt packet\_queue.c}. In reality, only
Netfilter id's of packets are stored in this queue (ns packets in the
ns case), while the real data packets are queued in kernel space by
the {\tt ip\_queue.o} Netfilter module. The packet queue implements
garbage collection of old packets and a queue limit. Parameters for
this functionality can be found and tweaked in {\tt packet\_queue.h}.

\section{Event handling}
The main loop of the AODV-UU implementation is based around a
non-blocking {\tt select()} call. This function will listen on a set
of file descriptors for readable characters. In the AODV-UU case,
there are the netlink socket used to communicate with the kernel's
netfilter functionality and one AODV socket for each interface
listening for packets on port 654.

\subsection{AODV-UU timer queue}
\index{timer\_queue.c} 

AODV-UU uses a timer queue to scheduale timeout events, i.e. routing
table entruies that are to expire, black-list sets to be emptied and
so on. This is implemented in {\tt timer\_queue.c} as list of timer
structs ordered by their time of expire. The timer stucts contain a
pointer to a function to call when the timer expires along with a
pointer to data to pass to that function. 

The queue is aged by passing the expire time of the first timer to the
main loops {\tt select()} call. The {\tt select()} is set to timeout
at this time and will do so unless there are information waiting to be
read on the sockets and file descriptors it is watching before the
timeout. In that case the information will be read and a new timeout
is scheduled for {\tt select()} with the remaining time until the
timeout.

\begin{thebibliography}
\small
\def\newblock{}

\bibitem{ns-2}
\textsc{The Network Simulator - ns-2:}
\newblock \url{http://www.isi.edu/nsnam/ns/}
\end{thebibliography}

\printindex

\end{document}
